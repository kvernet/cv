%%%%%%%%%%%%%%%%%%%%%%%%%%%%%%%%%%%%%%%%%%%%%%%%%%%%%%%%%%%%%%%%%%%%%%%%%%%%%%%%%
%%%%%%%% Some useful definitions
%%%%%%%%%%%%%%%%%%%%%%%%%%%%%%%%%%%%%%%%%%%%%%%%%%%%%%%%%%%%%%%%%%%%%%%%%%%%%%%%%


\newcommand{\anr}{Agence Nationale de la Recherche}
\newcommand{\dire}[0]{Data Integration, Risk and Environment}
\newcommand{\lpc}[0]{Laboratoire de Physique de Clermont}
\newcommand{\lmv}{Laboratoire Magmas et Volcans}
\newcommand{\limos}{LIMOS}


\newcommand{\uca}[0]{Université Clermont Auvergne}
\newcommand{\ucbl}[0]{Université Claude Bernard Lyon 1}
\newcommand{\ueh}[0]{Université d'État d'Haïti}
\newcommand{\ipnl}[0]{IPNL(IP2I)}
\newcommand{\cern}[0]{CERN}
\newcommand{\aegis}{Antimatter Experiment gravity Interferometry and Spectroscopy}
\newcommand{\grpc}{Glass Resistive Plate Chamber}
\newcommand{\fds}{Faculté des Sciences}
\newcommand{\edsf}{Ecole Doctorale des Sciences Fondamentales}
\newcommand{\jrjc}{Journées de Rencontre des Jeunes Chercheurs}
\newcommand{\mers}{Ministry of Education and Scientific Research}

\newcommand{\comment}[1]{}


\documentclass[
	10pt,
]{FreemanCV}

\columnratio{0.46, 0.54}

\begin{document}

\begin{paracol}{2}

\parbox[][0.11\textheight][c]{\linewidth}{
	\centering
	
	{\sffamily\Huge Kinson VERNET}
}


\section{Professional skills}

\jobentry
	{}
	{}
	{}
	{}
	{\textbf{Research, Physics, Computer programming}}\\
	{	
	-- Simulation, Data analysis, Modelling\\
	-- Inversion problem, Monte Carlo, Linear programming\\
	-- Algorithms, Computer Programming\\
	-- Relational databases\\
	-- Web development and mobile applications
	}
\medskip


\section{Technical skills}

\jobentry
	{}
	{}
	{}
	{}
	{
	-- Simulation and modelling: Geant4, FreeCAD\\
	-- Data analysis: Python, ROOT CERN\\
	-- Programming languages: C, C++, C\#, Java\\
	-- Web development: HTML, CSS, JavaScript, PHP, MySQL\\
	-- Operating systems: Linux, Windows\\
	-- Other: Bash, Git, LaTeX, Lua, Markdown, SGE/Slurm
	}
\medskip


\section{Professional experience}

\jobentry
	{06/2023 -- 02/2024}
	{Full time}
	{Postdoc -- \uca}
	{}
	{\textbf{Analysis of data collected on volcanoes Masaya, Etna and
Vulcano in 2021 and 2022}}\\
	{
		During this postdoctoral research, I modelled the detector using FreeCAD and integrated it into the Geant4 simulation. Rust was used for radiation transport, and Python for data analysis.
	}
\medskip
\medskip

\jobentry
	{10/2019 -- 12/2022}
	{Full time}
	{PhD -- \uca}
	{}{}
	{\textbf{Density imaging of volcanoes using atmospheric muons}}\\
	{
		The primary objective was to develop a method for radiographing volcanoes to forecast volcanic hazards. The method takes into account various systematic effects that may impact volcano radiography. Ultimately, the method allows to radiography a volcano with a statistical uncertainty that does not surpass a threshold value and with optimal spatial resolution.
	}
\medskip
\medskip

\jobentry
	{Since 01/2019}
	{Full time}
	{Teacher -- \ueh}
	{}
	{\textbf{Algorithms and C/C++ programming}}\\
	{
		I teach algorithms and C programming (data structure, dynamic memory allocation) and object-oriented programming in C++. This experience has taught me how to pass on knowledge to people of different abilities and to be more tolerant.
	}
\medskip
\medskip

\jobentry
	{10/2014 -- 07/2016}
	{Partial time}
	{Teacher -- \unif}
	{}
	{\textbf{Mathematics, physics and linear programming}}\\
	{
		Functions, derivatives, integrals, matrices\\
		Ordinary differential equations\\
		Kinematics and newtonian dynamics\\
		Statistics, modeling concepts\\
		Simplex algorithm \& optimization problems
	}
\medskip
\medskip

\jobentry
	{02/2012 -- 07/2015}
	{Full time}
	{Software programmer -- \usjldd}
	{}
	{\textbf{Windows software design and development (C\#, MySQL)}}\\
	{
		I was in charge to develop Windows applications for use within the plant. That position allowed me to refine my skills in object-oriented programming and relational database management.
	}
\medskip
\medskip


\section{References}

Upon request


\switchcolumn

\parbox[top][0.11\textheight][c]{\linewidth}{
	\colorbox{shade}{
		\begin{supertabular}{@{\hspace{3pt}} p{0.05\linewidth} | p{0.775\linewidth}}
			%\raisebox{-1pt}{\faHome} & 133 Bvd Lafayette, 63000 Clermont-Ferrand, France\\
			%\raisebox{-1pt}{\faPhone} & (+33) 7 49 55 56 74\\
			\raisebox{-1pt}{\small\faEnvelope} & \href{mailto:kingvernet@yahoo.fr}{kingvernet@yahoo.fr}\\
			\raisebox{-1pt}{\small\faDesktop} & \href{https://kvernet.com}{https://kvernet.com}\\
			\raisebox{-1pt}{\faGithub} & \href{https://github.com/kvernet}{https://github.com/kvernet}\\
			\raisebox{-1pt}{\faLinkedinSquare} & \href{https://www.linkedin.com/in/kvernet}{https://www.linkedin.com/in/kvernet}\\
		\end{supertabular}
	}
	\vfill
}


\section{Certification} 

\begin{supertabular}{r l}
	\qualificationentry
		{2024}
		{Machine Learning}
		{}
		{(Un)Supervised Learning, Recommender S., Reinforcement L. (\href{https://coursera.org/share/47e6122bb5a1ad45a6fd45fe803dfaf7}{Link})}
		{\coursera}
\end{supertabular}


\section{Education} 

\begin{supertabular}{r l}
	\qualificationentry
		{2019 -- 2022}
		{PhD}
		{}
		{Particles, Interactions, Universe}
		{\uca}
	\qualificationentry
		{2016-2018}
		{Master I \& II}
		{First class honors}
		{Atoms, Molecules, Materials and Environment}
		{\ueh}
	\qualificationentry
		{2006 -- 2011}
		{Electro-mechanical Engineering}
		{}
		{\fds}
		{\ueh}
\end{supertabular}


\section{Research internship}
\jobentry
	{04/2019 -- 09/2019}
	{Full time}
	{\lpc}
	{}
	{\textbf{Radiography of volcanoes}}\\
	{
		During this internship, I studied systematic effects in volcano radiography and determined the optimal deployment site using backward Monte Carlo simulation.
	}
\medskip
\medskip

\jobentry
	{05/2018 -- 07/2018 \& 09/2018 -- 12/2018}
	{Full time}
	{\ipnl \,\& \cern}
	{}
	{\textbf{Antimatter detection}}\\
	{
		My main task was to optimize the detection of antimatter in the AEgIS (\aegis) experiment through simulation and data analysis. I was able to show coincidence in detection between our muon trajectograph and the existing scintillators.
	}


\section{Competition}

\begin{supertabular}{r l}	
	\tableentry{2019}{\textbf{PhD scholarship}}{}
	\tableentry{}{\textit{\uca}}{}
	\tableentry{}{}{}
\end{supertabular}


\section{Scientific communication}

\begin{supertabular}{r l}	
	\tableentry{Lecture}{Volcano imaging using atmospheric muons}{}
	\tableentry{}{JRJC -- 10/2021 (\href{https://indico.in2p3.fr/event/24590/contributions/100391}{Link to my contribution})}{}	
	\tableentry{Scientific days}{15 minutes scientific communication}{}
	\tableentry{}{Doctoral School -- 12/2021}{}	
	\tableentry{PhD seminar}{Volcano imaging using atmospheric muons}{}
	\tableentry{}{\lpc -- 02/2022}{}
\end{supertabular}


\section{Summer school}

\begin{supertabular}{r l}
	\tableentry{School of Statistics}{IN2P3 School of Statistics 2021}{}
	\tableentry{}{\href{https://indico.in2p3.fr/event/20220}{Indico link}}{}	
\end{supertabular}


\section{Research skills}

Scientific and methodological watch,
Critical analysis of scientific production,
Ability to accept criticism, humility demonstration, scientific doubt and ethics,
Reproducibility methods and reliable results,
Rigor, scientific integrity, traceability and validity of results,
Transfer, valorization and discussion of my research results,
Ability to work in a team,
Coordination,
Ability to adapt.


\section{Publications}
Valentin Niess, Kinson Vernet , Luca Terray (submitted). Goupil: A revertible Monte Carlo engine for low energy gamma rays.
\medskip
\newline
%Kinson Vernet et al. (en cours). A kernel-based approach to reconstruct volcano density with optimal spatial resolution and statistical uncertainty threshold.
%\medskip % Vertical whitespace
Kinson Vernet (2022). 3D Volcano Imaging Using Transmission Muography. \textit{JRJC 2021. Book of Proceedings. hal-03832762v1}.

\medskip

\end{paracol}

\end{document}
