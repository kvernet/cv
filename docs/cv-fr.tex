%%%%%%%%%%%%%%%%%%%%%%%%%%%%%%%%%%%%%%%%%%%%%%%%%%%%%%%%%%%%%%%%%%%%%%%%%%%%%%%%%
%%%%%%%% Some useful definitions
%%%%%%%%%%%%%%%%%%%%%%%%%%%%%%%%%%%%%%%%%%%%%%%%%%%%%%%%%%%%%%%%%%%%%%%%%%%%%%%%%


\newcommand{\anr}{Agence Nationale de la Recherche}
\newcommand{\dire}[0]{Data Integration, Risk and Environment}
\newcommand{\lpc}[0]{Laboratoire de Physique de Clermont}
\newcommand{\lmv}{Laboratoire Magmas et Volcans}
\newcommand{\limos}{LIMOS}


\newcommand{\uca}[0]{Universit\'e Clermont Auvergne}
\newcommand{\ucbl}[0]{Universit\'e Claude Bernard Lyon 1}
\newcommand{\ueh}[0]{Universit\'e d'\'Etat d'Haïti}
\newcommand{\unif}[0]{Universit\'e Fond'oies}
\newcommand{\usjldd}[0]{Usine sucrière Jean L\'eopold Dominique de Darbonne}
\newcommand{\ipnl}[0]{IPNL(IP2I)}
\newcommand{\cern}[0]{CERN}
\newcommand{\aegis}{Antihydrogen Experiment: Gravity, Interferometry, Spectroscopy}
\newcommand{\grpc}{Glass Resistive Plate Chamber}
\newcommand{\fds}{Facult\'e des Sciences}
\newcommand{\edsf}{\'Ecole Doctorale des Sciences Fondamentales}
\newcommand{\jrjc}{Journ\'ees de Rencontre des Jeunes Chercheurs}
\newcommand{\mers}{Ministry of Education and Scientific Research}

\newcommand{\comment}[1]{}


\documentclass[
	10pt,
]{FreemanCV}

\columnratio{0.46, 0.54}

\begin{document}

\begin{paracol}{2}

\parbox[][0.11\textheight][c]{\linewidth}{
	\centering
	{\sffamily\Huge Kinson VERNET}
}


\section{Comp\'etences m\'etier}

\jobentry
	{}
	{}
	{}
	{}
	{\textbf{Recherche, Physique, Informatique}}\\
	{	
	-- Mod\'elisation, Simulation, Analyse de donn\'ees\\
	-- M\'ethode Monte Carlo, Problème d’inversion\\
	-- Programmation lin\'eaire\\
	-- Algorithmique, Programmation informatique\\
	-- Base de donn\'ees relationnelles\\
	-- D\'eveloppement web et applications mobiles
	}
\medskip


\section{Comp\'etences techniques}

\jobentry
	{}
	{}
	{}
	{}
	{
	-- Simulation \& mod\'elisation: Geant4, FreeCAD\\
	-- Analyse de donn\'ees: Python, ROOT CERN\\
	-- Langages de programmation: C, C++, C\#, Java\\
	-- D\'eveloppement web: HTML, CSS, JavaScript, PHP, MySQL\\
	-- Systèmes d'exploitation: Linux, Windows\\
	-- Autres: Cmake, Bash, Git, LaTeX, Lua, Markdown, Slurm
	}
\medskip


\section{Exp\'erience professionnelle}

\jobentry
	{06/2023 -- 02/2024}
	{Temps plein}
	{Postdoc -- \uca}
	{}
	{\textbf{Analyse de donn\'ees recueillies sur les volcans Masaya, Etna et Vulcano 2021 et 2022}}\\
	{
	-- Développer des codes de simulation Monte Carlo\\
	-- Calibrer le détecteur à photons gamma\\
	-- Faire évoluer le modèle du détecteur\\
	-- Documenter la simulation\\
	-- Déployer la simulation sur le centre de calcul
	}
\medskip
\medskip

\jobentry
	{10/2019 -- 12/2022}
	{Temps plein}
	{Doctorat -- \uca}
	{}
	{\textbf{Imagerie de densit\'e des volcans par muographie}}\\
	{
	-- Simulations Monte Carlo\\
	-- Analyse de données\\
	-- Calibration et optimisation du télescope à muons\\
	-- Étude précise du bruit de fond\\
	-- Reconstruction de la densité du volcan\\
	-- Effets systématiques en radiographie des volcans
	}
\medskip
\medskip

\jobentry
	{Depuis 01/2019}
	{Temps plein}
	{Enseignant -- \ueh}
	{}
	{\textbf{Algorithmique \& programmation en C/C++}}\\
	{
	-- Algorithmique\\
	-- Introduction à la programmation en C\\
	-- Structure de données\\
	-- Allocation dynamique de la mémoire\\
	-- Programmation orientée objet (POO) en C++
	}
\medskip
\medskip

\jobentry
	{10/2014 -- 07/2016}
	{Temps partiel}
	{Enseignant -- \unif}
	{}
	{\textbf{Math\'ematiques, physique et programmation lin\'eaire}}\\
	{
	-- Fonctions, d\'eriv\'ees, int\'egrales, matrices\\
	-- \'Equations diff\'erentielles ordinaires\\
	-- Cin\'ematique et dynamique newtonienne\\
	-- Statistiques, concepts de mod\'elisation\\
	-- Algorithme de simplex et problèmes d’optimisation
	}
\medskip
\medskip

\jobentry
	{02/2012 -- 07/2015}
	{Temps plein}
	{Programmeur -- \usjldd}
	{}
	{\textbf{Conception et d\'eveloppement de logiciels Windows (C\#, MySQL)}}\\
	{
		Durant ce poste, j'ai d\'ev\'elopp\'e des applications Windows \`a utiliser dans toute l'usine. Ce poste m'a permis de parfaire mes comp\'etences en programmation orrient\'ee objet et en gestion de bases de donn\'ees relationnelles.
	}
\medskip
\medskip


\section{References}

Sur demande


\switchcolumn


\parbox[top][0.11\textheight][c]{\linewidth}{
	\colorbox{shade}{
		\begin{supertabular}{@{\hspace{3pt}} p{0.05\linewidth} | p{0.775\linewidth}}
			%\raisebox{-1pt}{\faHome} & address\\
			%\raisebox{-1pt}{\faPhone} & phone\\
			\raisebox{-1pt}{\small\faEnvelope} & \href{mailto:kingvernet@yahoo.fr}{kingvernet@yahoo.fr}\\
			\raisebox{-1pt}{\small\faDesktop} & \href{https://kvernet.com}{https://kvernet.com}\\
			\raisebox{-1pt}{\faGithub} & \href{https://github.com/kvernet}{https://github.com/kvernet}\\
			\raisebox{-1pt}{\faLinkedinSquare} & \href{https://www.linkedin.com/in/kvernet}{https://www.linkedin.com/in/kvernet}\\
		\end{supertabular}
	}
	\vfill
}


\section{Formation} 

\begin{supertabular}{r l}
	\qualificationentry
		{2019 -- 2022}
		{Doctorat}
		{}
		{Particules, Interactions, Univers}
		{\uca}
	\qualificationentry
		{2016-2018}
		{Master I \& II}
		{}
		{Atomes, mol\'ecules, mat\'eriaux et environnement}
		{\ueh}
	\qualificationentry
		{2006 -- 2011}
		{G\'enie \'electrom\'ecanique}
		{}
		{\fds}
		{\ueh}
\end{supertabular}


\section{Stages de recherche}
\jobentry
	{04/2019 -- 09/2019}
	{Temps plein}
	{\lpc}
	{}
	{\textbf{Radiographie des volcans}}\\
	{
		Pendant ce stage, j'ai \'etudi\'e quelques effets syst\'ematiques en radiographie des volcans. J'ai r\'eussi, par exemple, \`a d\'eterminer le site de d\'eploiement optimal par simulation Monte Carlo \`a rebours.
	}
\medskip
\medskip

\jobentry
	{05/2018 -- 07/2018 \& 09/2018 -- 12/2018}
	{Temps plein}
	{\ipnl \,\& \cern}
	{}
	{\textbf{D\'etection de l'antimatière}}\\
	{
		J'avais pour mission d'optimiser la d\'etection de l'antimati\`ere dans l'exp\'erience AEgIS (\aegis). J'ai r\'eussi \`a montrer des co\"incidences de d\'etection entre notre trajectographe \`a muons et les scintillateurs existants.
	}


\section{Concours}

\begin{supertabular}{r l}
	\tableentry{2019}{\textbf{Bourse de l'\edsf}}{}
	\tableentry{}{\textit{\uca}}{}
	\tableentry{}{}{}
\end{supertabular}


\section{Communication scientifique}

\begin{supertabular}{r l}
	\tableentry{Conf\'erence}{Imagerie de densit\'e des volcans}{}
	\tableentry{}{JRJC -- 10/2021 (\href{https://indico.in2p3.fr/event/24590/contributions/100391}{lien vers ma contribution})}{}
	\tableentry{Journ\'ees scientifiques}{15 minutes de communication scientifique}{}
	\tableentry{}{\'ecole doctorale -- 12/2021}{}
	\tableentry{S\'eminaire de doctorat}{Imagerie de densit\'e des volcans}{}
	\tableentry{}{\lpc -- 02/2022}{}	
\end{supertabular}


\section{Ecole d'\'et\'e}

\begin{supertabular}{r l}
	\tableentry{\'ecole de statistiques}{IN2P3 School of Statistics 2021}{}
	\tableentry{}{\href{https://indico.in2p3.fr/event/20220}{Lien indico}}{}
\end{supertabular}


\section{Comp\'etences en matière de recherche}

Veille scientifique et m\'ethodologique,
Analyse critique de production scientifique,
Capacit\'e d'accepter la critique, d\'emonstration d'humilit\'e, doute scientifique et \'ethique,
M\'ethodes de reproductibilit\'e et r\'esultats fiables,
Rigueur, int\'egrit\'e scientifique, traçabilit\'e et validit\'e des r\'esultats,
Transfert, valorisation et discussion des r\'esultats de mes recherches,
Capacit\'e à travailler en \'equipe,
Coordination,
Capacit\'e d'adaptation.


\section{Publications}
Valentin Niess, Kinson Vernet , Luca Terray (soumis). Goupil: A revertible Monte Carlo engine for low energy gamma rays.
\medskip
\newline
%Kinson Vernet et al. (en cours). A kernel-based approach to reconstruct volcano density with optimal spatial resolution and statistical uncertainty threshold.
%\medskip
Kinson Vernet (2022). 3D Volcano Imaging Using Transmission Muography. \textit{JRJC 2021. Book of Proceedings. hal-03832762v1}.
\medskip

\end{paracol}

\end{document}
